\documentclass{article}
\usepackage[utf8]{inputenc}
\usepackage{biblatex}
\usepackage{todonotes}

\addbibresource{references.bib}

\title{Language Agnostic Generative Chatbot using Personal Chatlog Data}
\author{Torjus Iveland \& Vegar Andreas Bergum}
\date{April 2019}

\begin{document}

\maketitle

\begin{abstract}

In this report, we describe the implementation of a simple chtbot system
using sequence-to-sequence model. we find that

\if
- What we researched.
- How we did it.
- What we found out.
\fi
\end{abstract}

\section{Introduction}

A chatbot is a computer system which can interact with an user through natural
languge. Because humans tend to prefer more human-like interfaces, chatbots can
be very useful in applications
such as customer support, education, and personal productivity systems like
Google Assistant.  This project concerns such chatbots which converse
specifically using the Norwwegian language.

\paragraph{}
Most research on chatbot systems concerns chatbots which converse in English.
However, Norwegian has a syntactic structure which differs from that of
English. Therefore, it is not guaranteed that this research automatically
applies to Norwegian chatbots as well. Additionally, further problems arise
from the fact that training data in Norwegian is not as abundant as for
English.

\paragraph{}
In this project, we explore Norwegian chatbots, with the goal of verifying that
conventional chatbot-creation techniques also can function adequately in
Norwegian. We especially want to verify that deep learning techniques such as
sequence-to-sequence models \cite{Cho2014} can be used to create
general-purpose Norwegian chatbots. Such models are usually used for machine
translation, but they have also proven efective in the field of chatbots. The
main objective of this project is therefore to implement a simple Norwegian
chatbot using a sequence-to-sequence model, which for example could be used as
a smalltalk module in another mode domain-specific chatbot.

\paragraph{}
We differ between retrieval-based and generative chatbots. A retrieval-based
system usually map an user input to a predefined intent, and then retrieve an
answer from a set of answers belonging to the detected intent. A generative
system does not rely on such predefined sets of answers.  Instead, they are
able to automatically generate an answer to the provided query. In this
project, we restrict ourselves to the latter kind of model. We also restrict
ourselves to an user-initiative only model, which means that the chatbot simply
responds with an answer to each user query.

\if
- Something about how we want to structure the rest of the report.  
- Fix that goal, motivation, question and hypothesis are not clear.  
- Attention \cite{Bahdanau2015}, BERT and similar when we hopefully begin using it.
\fi

\section{Method}

% I dette kapitlet skal du skrive om hvordan du har gått frem metodisk, og vise
% hvordan valg av design og metode egner seg til å svare på problemstillingen
% din.  Kapitlet må kunne gi svar på disse spørsmålene: Hvordan samlet du inn
% datamaterialet?  Hvordan behandlet du dataene du samlet inn?  Hvorfor valgte
% du disse metodene?  Hva er styrkene og svakehetene ved disse metodene?  Du
% skal også si noe om hvorfor du har gjort din undersøkelse på den måten du
% gjorde – og da peke på styrker og svakheter. I tillegg skal du drøfte etiske
% aspekter ved prosjektet. På den måten viser du at du har kommet frem til
% resultatene på en pålitelig og troverdig måte, men også at du er reflektert
% og kritisk overfor arbeidet du har gjort.  Husk også at du her, slik som i
% teorikapitlet, bare skal skrive om det metodiske som er relevant for din
% studie.



\if
 - Prøvd ut pre-existing modeller med forskjellige datasett
    - engelsk datasett som vi vet er av god kvalitet
    - forskjellige attention-modeller
 - Sammenlignet egne modeller og eget datasett mot dette
 - Ressursbegrensning (datakraft og tid) har begrenset omfanget på evaluering
 av forskjellige modeller
 - 

\fi

\section{Data}

Norwegian chatbot training data is not readily available, but relevant corpora
still exist.  For example, multiple spoken language corpora are available
through the CLARINO project \cite{clarino-about}. Examples of CLARINO
corpora which might prove useful for a chatbot project are the Big Brother
corpus \cite{clarino-bb} and NoTa-Oslo \cite{clarino-nota}, which both provide
spoken language annotations for he Norwegian language. However, both these
corpora have quite restrictive licenses and are only accessible online for
privacy reasons. Thus, the CLARINO corpora are not well suited for our chatbot
usage.

\paragraph{}
Another posible data source is to use exports of personal chatbot data from
various online platforms, For example, Facebook allows users to export their
private Messenger chat data. If all parties involved in a conversation consent,
such data can be used to train a chatbot.

\todo{skriv noe mer om hvordan vi bruker messenger data blabla}

\paragraph{}
In addition to using personal Facebook Messenger data, the Cornell Movie
Dialogs Corpus \cite{cornell-corpus} has been used for model evaluation. The
Cornell corpus consists of several hundreds of thousands utterances from movie
scripts that are freely available. This corpus is frequently highlighted in
chatbot models as a high quality data set.

\if
- Difficulties with chatbot data 
- We use chatbot data, want to check if it is possible to use real chatlot data
to create a language-agnostic bot 
\fi

\section{Implementation}

\if
- seq2seq 
- preprocessing of norwegian data - describe stack 
- teacher forcing
\fi

% Architecture \ Experimental Setup + Teacher forcing Future work: Bert/custom
% embeddings, embedding layer, bidirectional model, beam search

\section{Results}

% Å analysere gjør du ved å redegjøre, forklare og vurdere funnene dine.
% Analysedelen av oppgaven blir ofte kalt resultater, slik som i
% IMRoD-modellen.  I kvantitative studier vil du kanskje i tillegg til å
% presentere funnene skriflig, bruke figurer og tabeller for å gi leseren en
% oversikt og innsikt i hva du har gjort.  I empirisk baserte studier vil
% analysene handle om å beskrive og tolke. Mange vil ofte drøfte enkeltfunnene
% i dette kapitlet og ta for seg mer overordnede funn i drøftingskapitlet.

\section{Discussion / Evaluation}

% Du skal her drøfte resultatene dine og sette dem inn i en sammenheng. Å
% drøfte vil si å: sette ulike synspunkter, momenter, argumenter, faktorer og
% årsaker opp mot hverandre vurdere og sette dem opp mot hadily available, this
% is one of the primary challenges.  + Facebook Messenger data dump.  +
% Clarino: Tried, but restrictive licenses.  + Movie scripts. Copyright
% unclear.

\section{Related Work}

Multiple Norwegian chatbots already exist. However, these are usually systems
which first map the user query to a predefined intent, and then retrieve an
answer from a list of answers belonging to the intent. Such systems require a
fair amount of manual training, and lack the ability to independently formulate
an answer independently, based only on the question.  In this report, we
describe a generative chatbot which is based on an unsupervised model.


\section{Results}

% Å analysere gjør du ved å redegjøre, forklare og vurdere funnene dine.
% Analysedelen av oppgaven blir ofte kalt resultater, slik som i
% IMRoD-modellen.  I kvantitative studier vil du kanskje i tillegg til å
% presentere funnene skriflig, bruke figurer og tabeller for å gi leseren en
% oversikt og innsikt i hva du har gjort.  I empirisk baserte studier vil
% analysene handle om å beskrive og tolke. Mange vil ofte drøfte enkeltfunnene
% i dette kapitlet og ta for seg mer overordnede funn i drøftingskapitlet.

\section{Discussion / Evaluation}

% Du skal her drøfte resultatene dine og sette dem inn i en sammenheng. Å
% drøfte vil si å: sette ulike synspunkter, momenter, argumenter, faktorer og
% årsaker opp mot hverandre vurdere og sette dem opp mot hverandre. Finnes det
% flere ulike tolkninger av resultatene?

% heavily biased dataset bad training data, not so much training data simple
% model

% heavily biased dataset
% bad training data, not so much training data
% simple model

\section{Conclusions}

% Om avslutningen din skal være en konklusjon eller en oppsummering, avhenger
% av problemstillingen din. En konklusjon skal svare på problemstillingen, mens
% en oppsummering gjentar det viktigste fra oppgaven. Det er ikke uvanlig å
% velge en kombinasjon av de to, hvor du både oppsummerer oppgaven kort, men
% også svarer på problemstillingen.  Det er lurt å la avslutningen speile
% innledningen, ved å si hva du har gjort. Avslutningen bør også sette oppgaven
% din i et større perspektiv, og peke på hvilke muligheter du ser ut fra ditt
% prosjekt. Hvilke bidrag har din undersøke gitt til faget? Er det noe som
% burde blitt studert ytterligere? Slik tar du utgangspunkt i ditt eget
% prosjekt og peker på mulighet for oppfølging.

\printbibliography

\end{document}
