\documentclass{article}
\usepackage[utf8]{inputenc}
\usepackage{parskip}
\usepackage{biblatex}

\addbibresource{references.bib}

\title{TDT4310 --- Project Assignment}
\author{Torjus Iveland \& Vegar Andreas Bergum}
\date{April 2019}

\begin{document}

\maketitle

\section{Abstract}

+ What we researched.
+ How we did it.
+ What we found out.

\section{Introduction}

A chatbot is a computer system which can interact with an user through natural languge. Because
humans tend to prefer more human-like interfaces, chatbots can be very useful in applications
such as customer support, education, and personal productivity systems like Google Assistant.
This project concerns such chatbots which converse specifically using the Norwwegian language.

Most research on chatbot systems concerns chatbots which converse in English. However, Norwegian
has a syntactic structure which differs from that of English. Therefore, it is not guaranteed that
this research automatically applies to Norwegian chatbots as well. Additionally, further problems
arise from the fact that training data in Norwegian is not as abundant as for English.

In this project, we explore Norwegian chatbots, with the goal of verifying that conventional
chatbot-creation techniques also can function adequately in Norwegian. We especially want to
verify that deep learning techniques such as sequence-to-sequence models \cite{Cho2014} can be
used to create general-purpose Norwegian chatbots. Such models are usually used for machine
translation, but they have also proven efective in the field of chatbots. The main objective
of this project is therefore to implement a simple Norwegian chatbot using a sequence-to-sequence
model, which for example could be used as a smalltalk module in another mode domain-specific chatbot.

We differ between retrieval-based and generative chatbots. A retrieval-based system usually map
an user input to a predefined intent, and then retrieve an answer from a set of answers belonging
to the detected intent. A generative system does not rely on such predefined sets of answers.
Instead, they are able to automatically generate an answer to the provided query. In this project,
we restrict ourselves to the latter kind of model. We also restrict ourselves to an user-initiative
only model, which means that the chatbot simply responds with an answer to each user query.

+ Something about how we want to structure the rest of the report.
+ Fix that goal, motivation, question and hypothesis are not clear.
+ Attention \cite{Bahdanau2015}, BERT and similar when we hopefully begin using it.

\section{Method}

% I dette kapitlet skal du skrive om hvordan du har gått frem metodisk, og vise hvordan valg av design og metode egner seg til å svare på problemstillingen din.
% Kapitlet må kunne gi svar på disse spørsmålene:
% Hvordan samlet du inn datamaterialet?
% Hvordan behandlet du dataene du samlet inn?
% Hvorfor valgte du disse metodene?
% Hva er styrkene og svakehetene ved disse metodene?
% Du skal også si noe om hvorfor du har gjort din undersøkelse på den måten du gjorde – og da peke på styrker og svakheter. I tillegg skal du drøfte etiske aspekter ved prosjektet. På den måten viser du at du har kommet frem til resultatene på en pålitelig og troverdig måte, men også at du er reflektert og kritisk overfor arbeidet du har gjort.
% Husk også at du her, slik som i teorikapitlet, bare skal skrive om det metodiske som er relevant for din studie.

\section{Data}

% + Data in Norwegian is not readily available, this is one of the primary challenges.
% + Facebook Messenger data dump.
% + Movie scripts. Copyright unclear.

\section{Related Work}

Multiple Norwegian chatbots already exist. However, these are usually systems which first map
the user query to a predefined intent, and then retrieve an answer from a list of answers
belonging to the intent. Such systems require a fair amount of manual training, and lack
the ability to independently formulate an answer independently, based only on the question.
In this report, we describe a generative chatbot which is based on an unsupervised model.

\section{Implementation}

% Architecture \ Experimental Setup

\section{Results}

% Å analysere gjør du ved å redegjøre, forklare og vurdere funnene dine. Analysedelen av oppgaven blir ofte kalt resultater, slik som i IMRoD-modellen.
% I kvantitative studier vil du kanskje i tillegg til å presentere funnene skriflig, bruke figurer og tabeller for å gi leseren en oversikt og innsikt i hva du har gjort.
% I empirisk baserte studier vil analysene handle om å beskrive og tolke. Mange vil ofte drøfte enkeltfunnene i dette kapitlet og ta for seg mer overordnede funn i drøftingskapitlet.

\section{Discussion / Evaluation}

% Du skal her drøfte resultatene dine og sette dem inn i en sammenheng. Å drøfte vil si å:
% sette ulike synspunkter, momenter, argumenter, faktorer og årsaker opp mot hverandre
% vurdere og sette dem opp mot hverandre. Finnes det flere ulike tolkninger av resultatene?

\section{Conclusions}

% Om avslutningen din skal være en konklusjon eller en oppsummering, avhenger av problemstillingen din. En konklusjon skal svare på problemstillingen, mens en oppsummering gjentar det viktigste fra oppgaven. Det er ikke uvanlig å velge en kombinasjon av de to, hvor du både oppsummerer oppgaven kort, men også svarer på problemstillingen.
% Det er lurt å la avslutningen speile innledningen, ved å si hva du har gjort. Avslutningen bør også sette oppgaven din i et større perspektiv, og peke på hvilke muligheter du ser ut fra ditt prosjekt. Hvilke bidrag har din undersøke gitt til faget? Er det noe som burde blitt studert ytterligere? Slik tar du utgangspunkt i ditt eget prosjekt og peker på mulighet for oppfølging.

\printbibliography

\end{document}
